\documentclass[12pt,a4paper]{article}
\usepackage{essv}
% veraltet: \usepackage{times}
\usepackage{mathptmx}
\usepackage[scaled=.90]{helvet}
\usepackage{courier}

\usepackage{amsmath}
\usepackage{amsfonts}
\usepackage{amssymb}
\usepackage{slashbox}
\usepackage{enumerate}
\usepackage[]{algorithm2e}
%
\pagestyle{empty}
\usepackage{ngerman}
\selectlanguage{ngerman}
\usepackage{bibgerm}
\usepackage[ansinew]{inputenc}

%
\usepackage{ifpdf}
\ifpdf
  \usepackage[pdftex]{graphicx}
  \DeclareGraphicsExtensions{.pdf,.jpg,.png,.mps}
  \usepackage[pdftex]{color}
  \pdfcompresslevel=9
\else
  \usepackage{graphicx}
  \DeclareGraphicsExtensions{.pstex,.eps,.mps}
  \usepackage[dvips]{color}
\fi
\usepackage{epstopdf}
\usepackage{pgfplots}
\usepackage{pgfplotstable}
\usetikzlibrary{patterns}

% Grafikpfade
\graphicspath{{pictures/}}
%
\begin{document}
\title{
  UCUI - Nutzeridentifikation Arbeitspaket 3.5 
}
\author{
  Peter Ge\ss ler\\[6pt]
  Brandenburgische Technische Universit\"at\\
  gessler\_peter@outlook.de
}
\maketitle

\thispagestyle{empty}
\begin{abstract}
  Der vorliegende Report beinhaltet die Beschreibung und Auswertung einer
  textunabh\"angigen Nutzeridentifikation. Diese ist Gegenstand des
  Arbeitspaketes 3.5 im Projekt \textsl{Universal Cognitive User Interface}
  (UCUI). Mit dieser Untersuchung sollte nachgewiesen werden, inwiefern sich
  die vom \textsl{dLabPro} verwendeten \textsl{primary feature vectors (pfv)} zur
  textunabh\"angigen Nutzeridentifikation und -verifikation eignen. Jede Person,
  die zur Bedienung des Systems berechtigt ist, wird
  durch ein \textsl{Gaussian Mixture Model} (GMM) repr\"asentiert. F\"ur die
  R\"uckweisung unberechtigter Sprecher dient ein \textsl{Universal Background
  Model} (UBM).  Als Datenbasis diente der Verbmobil Korpus und die
  Datenbasis C-VAU \cite{2}.
\end{abstract}

\section{Untersuchungen - Verbmobil Datenbasis - pfv-Merkmale}

\subsection*{Aufbau}
Der Versuchsaufbau beinhaltete jeweils $10$ berechtigte und nicht
berechtigte Sprecher. Aufgrund der Anonymisierung aller Daten innerhalb des Verbmobil Korpus
kann keine Aussage hinsichtlich der Geschlechterverteilung getroffen werden. Die
Merkmalanalyse der Sprachdaten erfolgte mit den \textsl{default}-Einstellungen des
Spracherkenners. Dieser erzeugt \textsl{pfv}-Merkmale mit $30$ Dimensionen anhand
einer vordefinierten \textsl{MEL-Filterbank}. Die Auswahl der Sprachdaten pro
Sprecher fand sitzungs- und dialog\"ubergreifend statt. Es wurden jedoch nur
Sprachdaten von Sprechern ausgew\"ahlt, die mittels Headset aufgenommen wurden.
Jedes Sprechermodell (GMM) ist aus $10$ zuf\"allig gezogenen Spracheingaben
(Java-Skript mit \textsl{Random}-Funktion zur Auswahl der Daten) pro
berechtigtem Sprecher generiert worden. F\"ur die R\"uckweisung unberechtigter
Sprecher existiert ein 11. GMM. Das Training des UBM erfolgte mit den
ausgew\"ahlten Sprachdaten aller berechtigten Sprecher. Die einzelnen
Sprechermodelle wurden anschlie�end durch den \textsl{EM-Algorithmus} optimiert.
In den folgenden Experimenten konnte der Klassifikator zu jeder �u�erung aus dem
Testdatensatz genau ein Sprechermodell von den $11$ vorhandenen ausw�hlen.

\subsection*{Experiment 1.1 - geschlossene Nutzeridentifikation/-verifikation}
Mit der geschlossenen Nutzeridentifikation ist zun\"achst die Leistungsf\"ahigkeit
der einzelnen Sprechermodelle gegen\"uber dem UBM getestet worden. Daf\"ur
dienten zum Test jeweils $20$ zuf\"allig gezogene Sprachdaten pro
berechtigtem Sprecher. Es stellte sich heraus, dass die Verwendung eines GMM's
mit zwei Normalverteilungen und einem Iterationsschritt mit $94_{-4.2}^{+2.9}~\%$ das
beste Ergebnis erzielte. Es existiert jedoch kein signifikanter Unterschied
hinsichtlich eines GMM's mit einer Normalverteilung ohne Iteration
$93.5_{-4.4}^{+3.0}~\%$.

\subsection*{Experiment 1.2 - R\"uckweisung nicht berechtigter Sprecher}
Ziel des Experiments ist die R\"uckweisung durch den Spracherkenner aller
Sprachdaten. Diese stammen von nicht berechtigten Sprechern und sind dem UBM
zugeordnet. Mit der Auswahl des UBM Sprechermodells erfolgt im konkreten System
eine Zur\"uckweisung der Anweisung. F\"ur das Training sind $20$ Sprachdaten pro
nicht berechtigtem Sprecher verwendet worden. Die Auswahl des UBM erfolgte in
$64.0_{-7.1}^{+6.6}~\%$ aller F\"alle mit zwei Normalverteilungen pro
Sprechermodell und ohne Iteration. Jede h\"ohere Ordnung der Normalverteilungen
erzeugte eine rapide Verschlechterung des Erkennungsergebnisses.

\subsection*{Experiment 1.3 - offene Nutzeridentifikation/-verifikation}
F\"ur die offene Nutzeridentifikation/-verifikation sind alle Testdaten der
berechtigten und nicht berechtigten Personen verwendet worden. Dieses Szenario
entspricht dem m\"oglichen Anwendungsfall am ehesten. Mit $74.5_{-4.5}^{+4.1}~\%$
erzielte die Konfiguration -- zwei Normalverteilungen ohne Iteration -- das
h\"ochste Erkennungsergebnis. Die Verwendung einer einfachen Normalverteilung ohne
Iteration ist ebenfalls m\"oglich $74.0_{-4.5}^{+4.1}~\%$

\subsection*{Auswertung - Verbmobil}

\pgfplotstableread[col sep=comma,header=false]{
E11,94,0,0,6
E12,0,36,64,0
E13,48.57,19.2,27.0,1.67
}\data

\pgfplotstablecreatecol[
create col/expr={
    \thisrow{1} + \thisrow{2} + \thisrow{3} + \thisrow{4} 
}
]{sum}{\data}
\pgfplotsset{
percentage plot/.style={
    point meta=explicit,
every node near coord/.append style={
    align=center,
    text width=1cm
},
    nodes near coords={
    \pgfmathtruncatemacro\iszero{\originalvalue==0}
    \ifnum\iszero=0
        \pgfmathprintnumber{\originalvalue}$\,\%$\\ \pgfmathprintnumber[fixed zerofill,precision=1]{\pgfplotspointmeta}
    \fi},
nodes near coords align=vertical,
    yticklabel=\pgfmathprintnumber{\tick}\,$\%$,
    ymin=0,
    ymax=100,
    enlarge y limits={upper,value=0},
visualization depends on={y \as \originalvalue}
},
percentage series/.style={
    table/y expr=\thisrow{#1},table/meta=#1
}
}
\vspace{0.2cm}
\begin{tikzpicture}
\begin{axis}[
title style={at={(0.5,-0.15)},anchor=north,yshift=0.0},
title=Auswertung der Nutzeridentifikations-Experimente  mit Verbmobil., axis on
top, width=14cm,
height=12cm,
ylabel=Ergebnis in Prozent,
xlabel=Experimente,
percentage plot,
ybar=0pt,
bar width=0.75cm,
enlarge x limits=0.25,
symbolic x coords={E11, E12, E13},
xtick=data
]
\addplot table [percentage series=1] {\data};
\addplot table [percentage series=2] {\data};
\addplot table [percentage series=3] {\data};
\addplot table [percentage series=4] {\data};
\legend{korrekt angenommen, falsch angenommen, korrekt abgewiesen, falsch
abgewiesen}
\end{axis}
\end{tikzpicture}

Die Erkennungsergebnisse unter der Verwendung von pfv-Merkmalen und GMMs sowie
einem Maximum-Likelihood Klassifikator sind durchaus beachtenswert. Das
Verfahren eignet sich vor allem f\"ur die
\textsl{closed}-Nutzeridentifikation/-verifikation. Selbst bei spezifischen
Merkmalen f\"ur die Nutzeridentifikation liegen die Erkennungsquoten im Bereich
$96 - 98 \%$. Probleme bestehen jedoch in der Zur\"uckweisung nicht berechtigter
Sprecher (siehe Experiment 1.2).
Durch die Anonymisierung aller Sprachdaten konnte nicht bestimmt werden, ob ein
Geschlecht besonders h\"aufig zur\"uckgewiesen oder falsch zugeordnet wird. Des
Weiteren kann nicht der Einfluss sprachlicher Aspekte ermittelt werden, weil die
Aussagen der einzelnen Sprecher unterschiedlich sind. F\"ur weiterf\"uhrende
Erkenntnisse hinsichtlich der Verwendung von pfv-Merkmalen zur
textunabh\"angigen Nutzeridentifikation/-verifikation dienten Untersuchungen
mit der C-VAU Datenbasis.

\section{Untersuchungen - C-VAU Datenbasis pfv-Merkmale}

\subsection*{Aufbau}
Der C-VAU Korpus ist ausf\"uhrlich in dem Beitrag \cite{2} erl\"autert. F\"ur die
Nutzeridentifikation/-verifikation sind die Probanden in $10$ berechtigte und
$10$ nicht berechtigte Sprecher unterteilt worden. Es handelt sich dabei um $6$
berechtigte M\"anner und $4$ berechtigte Frauen sowie $5$ nicht berechtigte M\"anner
und $5$ nicht berechtigte Frauen. Zum Training der GMM Sprechermodelle sind pro
berechtigtem Sprecher $10$ Sprachdaten gezogen, die jeweils identisch zueinander
sind (berechtigte Sprecher $1$ bis $10$ haben denselben Satz gesagt). 
Alle Aufnahmen einer Person erfolgten innerhalb einer Sitzung. Zur
Aufnahme der Sprachdaten diente ein \textsl{Rode NT-1 A} Mikrofon sowie eine
\textsl{Focusrite Scarlett 8i6}. Weitere Details zu den Aufnahmen sind in \ldots
beschrieben. Das UBM ist ebenfalls aus allen verwendeten Sprachdaten der
berechtigten Sprecher generiert und kann neben den $10$ Sprechermodellen vom
Klassifikator ausgew�hlt werden.

\subsection*{Experiment 2.1 - geschlossene Nutzeridentifikation/-verifikation}
F\"ur die geschlossene Nutzeridentifikation/-verifikation wurden pro
berechtigtem Sprecher $10$ Sprachaufnahmen ausgew\"ahlt. Es sollte ebenfalls wie
in Experiment 1.1 die Leistungsf\"ahigkeit der einzelnen Sprechermodelle
und des Klassifikators gegen\"uber dem UBM getestet werden.

Die Zuordnungen der einzelnen Sprach\"au�erungen eines berechtigten
Sprechers zu seinem Sprechermodell erfolgte zu $100.0_{-0.0}^{+0.0}~\%$
korrekt. Dies trifft sowohl bei der Verwendung einer Normalverteilung oder zwei
Normalverteilungen pro Sprechermodell zu. Der EM-Algorithmus f\"uhrt hingegen zu
einer Verschlechterung des Erkennungsergebnisses.

\subsection*{Experiment 2.2 - R\"uckweisung nicht berechtigter Sprecher 1}
Im Gegensatz zum Experiment 2.1 \"au\ss erten die nicht berechtigten Personen in
2.2 genau dieselben S\"atze, welche die berechtigten Sprecher zur Generierung ihrer
Sprechermodelle verwendeten. Es l\"asst sich somit eine Aussage zur Robustheit
der pfv-Merkmale hinsichtlich der sprecherspezifischen Eigenschaften treffen. F\"ur
eine ideale Nutzeridentifikation/-verifikation sollte eine Zuweisung aller
Testdaten zum UBM stattfinden.
Die Auswahl des UBM erfolgte vom Klassifikator in $69.0_{-10.0}^{+8.9}~\%$ aller
F\"alle bei einer Normalverteilung pro Sprechermodell. Zwei nicht berechtigte weibliche
Probandinnen wurden \"uberwiegend falsch zugeordnet (zu zwei berechtigten
weiblichen Probandinnen).

\subsection*{Experiment 2.3 - R\"uckweisung nicht berechtigter Sprecher 2}
Ein direkter Vergleich zu Experiment 1.2 erfolgte durch die Verwendung von
Sprachdaten, deren S\"atze nicht identisch mit den der Trainingsdaten aller
berechtigten Sprecher waren. Pro nicht berechtigtem Sprecher enth\"alt der
Testdatensatz $10$ Sprachaufnahmen.
Mit $65.0_{-10.2}^{+9.3}~\%$ konnte das h\"ochste Ergebnis bei der Verwendung
einer Normalverteilung pro Sprecher erreicht werden. Wie bereits in Experiment
2.2 sind die beiden nicht berechtigten weiblichen Probandinnen zwei berechtigten
zugeordnet worden. Des Weiteren ordnete der Klassifikator ein nicht
berechtigten m\"anlichen Proband f\"alschlicherweise abwechselnd zwei
berechtigten m\"annlichen Probanden zu.

\subsection*{Experiment 2.4 - offene Nutzeridentifikation/-verifikation}
Experiment 2.4 schlie�t die Untersuchung vorerst ab und bietet
einen ersten Eindruck hinsichtlich des aufgestellten Anwendungsszenarios.
Daf\"ur dienten jeweils $10$ \"Au�erungen pro berechtigtem sowie nicht berechtigtem Sprecher.
Bei der Verwendung von Sprach\"au�erungen berechtigter und nicht berechtigter
Sprecher konnte der Klassifikator mit einer Normalverteilung pro Sprechermodell
eine Erkennungsquote von $82.5_{-6.0}^{+5.0}~\%$ erreichen.

\subsection*{Auswertung - C-VAU mit pfv-Merkmale}

\pgfplotstableread[col sep=comma,header=false]{
E21,100,0,0,0
E22,0,31,69,0
E23,0,35,65,0
E24, 50, 17.5,32.5,0
}\data

\pgfplotstablecreatecol[
create col/expr={
    \thisrow{1} + \thisrow{2} + \thisrow{3} + \thisrow{4} 
}
]{sum}{\data}
\pgfplotsset{
percentage plot/.style={
    point meta=explicit,
every node near coord/.append style={
    align=center,
    text width=1cm
},
    nodes near coords={
    \pgfmathtruncatemacro\iszero{\originalvalue==0}
    \ifnum\iszero=0
        \pgfmathprintnumber{\originalvalue}$\,\%$\\ \pgfmathprintnumber[fixed zerofill,precision=1]{\pgfplotspointmeta}
    \fi},
nodes near coords align=vertical,
    yticklabel=\pgfmathprintnumber{\tick}\,$\%$,
    ymin=0,
    ymax=100,
    enlarge y limits={upper,value=0},
visualization depends on={y \as \originalvalue}
},
percentage series/.style={
    table/y expr=\thisrow{#1},table/meta=#1
}
}
\vspace{0.2cm}
\begin{tikzpicture}
\begin{axis}[
title style={at={(0.5,-0.15)},anchor=north,yshift=0.0},
title=Auswertung der Experimente  mit C-Vau und pfv-Merkmalen., axis on
top, width=14cm,
height=12cm,
ylabel=Ergebnis in Prozent,
xlabel=Experimente,
percentage plot,
ybar=0pt,
bar width=0.75cm,
enlarge x limits=0.25,
symbolic x coords={E21, E22, E23, E24},
xtick=data
]
\addplot table [percentage series=1] {\data};
\addplot table [percentage series=2] {\data};
\addplot table [percentage series=3] {\data};
\addplot table [percentage series=4] {\data};
\legend{korrekt angenommen, falsch angenommen, korrekt abgewiesen, falsch
abgewiesen}
\end{axis}
\end{tikzpicture}

Durch Experiment 2.4 konnte gezeigt werden, dass sich pfv-Merkmale
durchaus f\"ur die textunabh\"angige m\"annliche Nutzeridentifikation/-verifikation
eignen. Da ich die in Experiment 2.3 und 2.4 genannten Probandinnen
pers\"onlich kenne, trifft nach subjektiver Einsch\"atzung durchaus die Aussage
zu, dass sich jeweils die nicht berechtigte und berechtigte Probandin \glqq gleich anh\"oren\grqq.

\section{Bewertung der Untersuchungen}
pfv-Merkmale lassen sich unter den gegebenen Bedingungen durchaus f�r die
Nutzeridentifikation einsetzen. Es sollte jedoch darauf geachtet werden, dass
die Anzahl der Sprechermodelle m�glichst gering ist und im Idealfall einer
\textsl{closed}-Nutzeridentifikation/-verifikation mit UBM entspricht. Dies
kann zum Beispiel bei Heimautomatisierungssystemen der Fall sein, bei dem jeder
berechtigte Sprecher nur eine Teilmenge von Aktionen ausf�hren darf.
pfv-Merkmale sind keine typischen Nutzeridentifikations/-verifikations
Merkmale wie zum Beispiel \textsl{Mel Frequency Cepstral Coefficients} (MFCC).
Des Weiteren repr�sentieren sie nicht sprecherspezifische Eigenschaften 1. oder
2. Ordnung. Untersuchung diesbez�glich erfolgten vorerst nicht. Neben der
falschen Klassifikation in Experiment 2.3 und 2.4 ist die Robustheit der
pfv-Merkmale bei sitzungs�bergreifender Klassifikation fraglich. Dieses Szenario
sollte jedoch erst mit einem konkreten Prototyp getestet werden. Neben den
bereits genannten Faktoren ist ebenfalls der Einfluss aller zum Training
ausgew�hlten Sprachdaten vorerst nicht messbar. Der Einsatz von pfv-Merkmalen in
\textbf{sicherheitskritischen Systemen} mit einer gro\ss en Anzahl m\"annlicher
und weiblicher Sprecher ist somit \underline{\textbf{nicht zu empfehlen}}.
Weitere Experimente hinsichtlich des Einflusses der zum Training verwendeten Sprachdaten, der
fehlerhaften Zuweisung weiblicher nicht berechtigter Probandinnen und der
Verwendung von sfv-Merkmalen zur Nutzeridentifikation/-verifikation stehen
noch aus. Zur Eliminierung der Kanaleigenschaften wird jedoch bereits jetzt die
Anwendung des \textsl{Mittelwertsubtraktion} (MS) nach \cite{1} empfohlen.

\section{Erweiterte Untersuchung sfv-Merkmale C-VAU Datenbasis}
Neben den pfv-Merkmalen sollte anschlie\ss end stichpunktartig die
Leistungsf\"ahigkeit der sfv-Merkmale, welche prim\"ar zur Spracherkennung
dienen, f�r die textunabh\"angige Nutzeridentifikation bzw. -verifikation
getesten werden. Dazu sind die Experimente 2.1 - 2.4 mit den sfv-Merkmalen
wiederholt worden.

\subsection*{Experiment 3.1 - geschlossene Nutzeridentifikation/-verifikation}
Bereits die geschlossene Nutzeridentifikation weist mit den sfv-Merkmalen
geringe Abweichungen ($99.0_{-4.4}^{+1.0}~\%$) gegen\"uber
dem Klassifikationsergebnis mit pfv-Merkmalen auf. Hierbei handelt es sich um
die fehlerhafte Auswahl des UBM.

\subsection*{Experiment 3.2 - R\"uckweisung nicht berechtigter Sprecher 1}
Bei der R\"uckweisung nicht berechtigter Sprecher kam es zu einer minimalen
Verbesserung ($72.0_{-9.9}^{+8.5}~\%$) des Klassifikationsergebnis gegen�ber der
Klassifikation mit pfv-Merkmalen $69.0_{-10.0}^{+8.9}~\%$. Im Gegensatz zur
Klassifikation mit pfv-Merkmalen sind die False Negative
Erkennungen nicht auf zwei Probandinnen begrenzt sonder liegen verstreut vor.

\subsection*{Experiment 3.3 - R\"uckweisung nicht berechtigter Sprecher 2}
Bei der textunabh\"angigen Nutzeridentifikation/-verifikation mit S\"atzen die
nicht zum Training der Sprechermodelle verwendet worden, erzielen die
sfv-Merkmale mit $72.0_{-9.9}^{+8.5}~\%$ eine Erkennungsquote die numerisch
$7~\%$ h\"oher ist als die der pfv-Merkmale ($65.0_{-10.2}^{+9.3}~\%$).
Die False Negative Erkennungen liegen jedoch ebenfalls verstreut vor und k�nnen
nicht auf bestimmte nutzerspezifische Eigenschaften zur\"uckgef\"uhrt werden.

\subsection*{Experiment 3.4 - offene Nutzeridentifikation/-verifikation}
Mit $85.5_{-5.7}^{+4.6}~\%$ liegt das Klassifikationsergebnis der sfv-Merkmale
nur minimal �ber dem der pfv-Merkmale mit $82.5_{-6.0}^{+5.0}~\%$. Wegen der
\"Uberlappung des Konfidenzintervalls kann jedoch von einer nicht vorhandenen
Signifikanz ausgegangen werden.

\subsection*{Auswertung C-VAU mit sfv-Merkmale}
\pgfplotstableread[col sep=comma,header=false]{
E41,99,0,0,1
E42,0,28,72,0
E43,0,28,72,0
E44, 49.5,14.5,35.5,0.5
}\data

\pgfplotstablecreatecol[
create col/expr={
    \thisrow{1} + \thisrow{2} + \thisrow{3} + \thisrow{4} 
}
]{sum}{\data}
\pgfplotsset{
percentage plot/.style={
    point meta=explicit,
every node near coord/.append style={
    align=center,
    text width=1cm
},
    nodes near coords={
    \pgfmathtruncatemacro\iszero{\originalvalue==0}
    \ifnum\iszero=0
        \pgfmathprintnumber{\originalvalue}$\,\%$\\ \pgfmathprintnumber[fixed zerofill,precision=1]{\pgfplotspointmeta}
    \fi},
nodes near coords align=vertical,
    yticklabel=\pgfmathprintnumber{\tick}\,$\%$,
    ymin=0,
    ymax=100,
    enlarge y limits={upper,value=0},
visualization depends on={y \as \originalvalue}
},
percentage series/.style={
    table/y expr=\thisrow{#1},table/meta=#1
}
}
\vspace{0.2cm}
\begin{tikzpicture}
\begin{axis}[
title style={at={(0.5,-0.15)},anchor=north,yshift=0.0},
title=Auswertung der Experimente  mit C-Vau und sfv-Merkmalen., axis on
top, width=14cm,
height=14cm,
ylabel=Ergebnis in Prozent,
xlabel=Experimente,
percentage plot,
ybar=0pt,
bar width=0.75cm,
enlarge x limits=0.25,
symbolic x coords={E41, E42, E43, E44},
xtick=data
]
\addplot table [percentage series=1] {\data};
\addplot table [percentage series=2] {\data};
\addplot table [percentage series=3] {\data};
\addplot table [percentage series=4] {\data};
\legend{korrekt angenommen, falsch angenommen, korrekt abgewiesen, falsch
abgewiesen}
\end{axis}
\end{tikzpicture}


\section{Bewertung der Untersuchungen mit sfv-Merkmalen}
Die Erkennungsquoten der Experimentdurchf�hrungen mit sfv-Merkmalen liegen
\"uber denen der pfv-Merkmale. Jedoch k\"onnen die vorhandenen False Negative
und False Positive bei einer detaillierten Betrachtung nicht auf nutzerspezifsche
Eigenschaften zur\"uckgef\"uhrt werden. Dementsprechend ist eine Verwendung der
sfv-Merkmale zur Nutzeridentifikation/-verifikation unter den gegebenen
Bedingungen nicht als sinnvoll zu erachten.

% BibTeX standard bibliography style for use with `bibgerman.sty'
% *.bib language field: language = "german" | "USenglish" | "english"
\bibliographystyle{gerabbrv}
\bibliography{ucui-vau}
%
% ABBRVDIN.BST wurde entwickelt aus Oren Patashnik's BibTeX standard
% bibliography style `abbrv'. Eine vorgegebene Literaturdatenbank laesst sich
% somit beliebig nach US- oder deutscher DIN 1505-Zitierkonvention verarbeiten.
% \bibliographystyle{abbrvdin}
%
\end{document}